\chapter{Introduction}
%%%% \newline

%% what is infinite matrices %%%
An infinite matrix is a matrix of infinite dimensions in which both the number of rows and columns are infinite. Infinite matrices, the forerunner and a primary constituent of many branches of classical mathematics (infinite quadratic forms, integral equations, differential equations, etc.) and the modern operator theory, are revisited to demonstrate its profound influence on the development of many branches of mathematics, classical and contemporary, replete with applications.
\newline

%%% history of infinite matrices %%%%%
According to Michael Bernkopf~\cite{History-infinite-matrics}, the history of a general theory of infinite matrices begins with Henri Poincare in 1884. After Poincare, Helge Von Koch was the next to take up the study, and by 1893, he had proved most of the theorems about infinite matrices and their determinants. Later in 1906, a tremendous impulse was given to the subject when David Hilbert used infinite quadratic forms, which are equivalent to infinite matrices, to solve the integral equation \[f(s) = \phi(s) + \lambda \int_{a}^{b} K(s, t) \phi(t) \, dt.\]
%Some authors used infinite matrices and determinants without logical justification; their purpose was to provide a rigorous basis.  His followers took up Hilbert's ideas- Erhard Schmidt, Ernest Hellinger, and Otto Toeplitz, among others - and within a few years, many of the theorems fundamental to the theory of more abstract operators had been discovered, although they were couched in special matrix terms. Finally, in 1929, he used an effective tool for studying operators on function spaces; instead, he demonstrated that an abstract approach was preferable.

According to P. N. Shivakumar and K. C. Shivakumar~\cite{shivakumar1972diagonally}, applications of matrices are found in most scientific fields. In every branch of physics, including classical mechanics, optics, electromagnetism, quantum mechanics, etc, they are used to study physical phenomena, such as the motion of rigid bodies. Computer graphics are used to manipulate 3D models and project them onto a 2-dimensional screen. In probability theory and statistics, stochastic matrices are used to describe sets of probabilities. Matrix calculus generalizes classical analytical notions such as derivatives and exponentials to higher dimensions. A significant branch of numerical analysis is devoted to developing efficient algorithms for matrix computations, a subject that is centuries old and an expanding area of research. 

This project endeavors to elucidate the fundamental principles governing infinite matrices, with a particular focus on their determinant properties. Through an exploration of parallels between finite and infinite matrices, key concepts such as eigenvalues, eigenvectors, and solutions to systems of linear equations are examined in detail.

Chapter 2 introduces the matrix logarithm, emphasizing its practical implications in both real and complex domains, while also discussing various matrix-related definitions and fundamental mathematical structures such as groups, rings, and fields.

Chapter 3 delves deeply into determinant properties, addressing scenarios in both finite and infinite matrices, showcasing their practical utility in solving linear equations and exemplifying methodologies like Cramer's rule.

Chapter 4 extensively analyzes matrix inverses, providing definitions and illustrative examples predominantly within finite contexts, and extending the discussion to encompass infinite scenarios, including propositions on matrix invertibility.

Chapter 5 focuses on eigenvalues and eigenvectors, providing precise definitions and examples, highlighting the importance of employing the trace of the logarithm in determinant calculations, and exploring convergence behaviors through graphical representations of determinant errors.

Finally, Chapter 6 includes the algorithms and corresponding MATLAB source codes developed and used for approximating determinants of matrices, finding inverse matrices, and determining approximations for multiple eigenvalues using series expansion.

In this project, we abstain from delving into discussing fundamental theorems specific to infinite matrices. Instead, we offer an overview of properties inherent to finite matrices that find applicability and relevance within infinite matrix contexts.

